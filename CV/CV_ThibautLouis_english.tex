% Original author:
% Trey Hunner (http://www.treyhunner.com/)
%
% Important note:
% This template requires the resume.cls file to be in the same directory as the
% .tex file. The resume.cls file provides the resume style used for structuring the
% document.
%
%%%%%%%%%%%%%%%%%%%%%%%%%%%%%%%%%%%%%%%%%

%----------------------------------------------------------------------------------------
%	PACKAGES AND OTHER DOCUMENT CONFIGURATIONS
%----------------------------------------------------------------------------------------

\documentclass{resume} % Use the custom resume.cls style

\usepackage[left=0.75in,top=0.6in,right=0.75in,bottom=0.6in]{geometry} % Document margins
\usepackage[hidelinks]{hyperref} 
\setcounter{totalnumber}{8}
\usepackage[usenames,dvipsnames]{xcolor}

\newcommand{\TIB}[1]{\textcolor{blue}{#1}}

\name{\large{Thibaut Louis}} % Your name
\address{  Laboratoire de Physique des 2 Infinis Irène Joliot Curie, 15 Rue Georges Clemenceau 91400 Orsay  \\ 
louis@lal.in2p3.fr} % Your phone number and email

\begin{document}


\begin{rSection}{Research}
\vspace{-0.4cm}
\begin{table}[h]
%%%\centering
{\def\arraystretch{1.5}\tabcolsep=0pt
\begin{tabular}{p{0.15\linewidth}p{0.75\linewidth}}
  2017--present & CNRS permanent researcher, Laboratoire de Physique des 2 Infinis Irène Joliot Curie \\
  2015--2017 & Post-doctoral Research Fellow, Institut Astrophysique de Paris with Prof Joseph Silk \\
  2014--2015 & Post-doctoral Research Fellow, Oxford University with Prof Joanna Dunkley \\
\end{tabular}%
}
\end{table}
\end{rSection}
\vspace{-0.6cm}


\begin{rSection}{Education}
\vspace{-0.4cm}
\begin{table}[h]
%%%\centering
{\def\arraystretch{1.5}\tabcolsep=0pt
\begin{tabular}{p{0.15\linewidth}p{0.75\linewidth}}

2011--2014 & University of Oxford, Department of Astrophysics and Christ Church College;
PhD in Astrophysics;	Advisor: Joanna Dunkley \\
2010 & Master in Theoretical Physics, Ecole Normale Supérieure de Lyon \\

\end{tabular}%
}
\end{table}
\end{rSection}
\vspace{-0.6cm}

\begin{rSection}{Major Collaborations}
\vspace{-0.4cm}
\begin{table}[h]
%%%\centering
{\def\arraystretch{1.5}\tabcolsep=0pt
\begin{tabular}{p{0.15\linewidth}p{0.75\linewidth}}
2018--present & Lead of the power spectrum and likelihood group of Simons Observatory \\
2016--present & Member of the Simons Observatory collaboration \\
2011--present & Member of the ACT collaboration (ACT/ACTPol/AdvACT) \\
\end{tabular}%
}
\end{table}
\end{rSection}

\vspace{-0.6cm}

\begin{rSection}{Supervision of	students and postdocs}
\vspace{-0.4cm}

\begin{table}[h]
%%\centering
{\def\arraystretch{1.5}\tabcolsep=0pt
\begin{tabular}{p{0.15\linewidth}p{0.75\linewidth}}
Post-doc  & Lukas Hergt (2024-), Hiroaki Imada (2018--2020) \\
Thesis  & Merry Duparc (2024-),  Adrien La Posta (2020--2023), I acted as co-supervisor for Sylvain Vanneste (2016--2019) \\
Undergrad   & Adam Soussana (2018), Magdy Morshed (2018), Marie Arbogast (2025)
\end{tabular}%
}
\end{table}
\end{rSection}

\vspace{-0.6cm}

\begin{rSection}{Public codes}
All the codes I have written are in the public domain, some of them can be accessed on my \href{https://github.com/thibautlouis}{\TIB{github}}, others on the Simons Observatory \href{https://github.com/simonobs}{\TIB{github}}. Here is an non exhaustive list of codes for which I have important contributions 
\begin{enumerate}
\item I am the main developper of \href{https://github.com/simonsobs/pspy}{\TIB{pspy}}, a public code allowing to estimate power spectra (and pure power spectra) of angular maps with Healpix and CAR pixellisation as well as their associated analytical covariance matrices. The code also implements the Toeplitz approximation that we have proposed in \href{https://ui.adsabs.harvard.edu/abs/2020PhRvD.102l3538L/abstract}{\TIB{this paper}}.
\item I am  the main developper of \href{https://github.com/simonsobs/PSpipe}{\TIB{PSpipe}}, a pipeline  allowing to reproduce the ACT DR6 and Planck results. 
\item I am one of the main developper of \href{https://github.com/simonsobs/mflike}{\TIB{mflike}}, which implements the multifrequency likelihood of the forthcoming Simons Observatory experiment. 

\end{enumerate}

\end{rSection}


\begin{rSection}{Publications}
At the time of writing this CV, I have authored 99 publications, a list can be accessed on \href{https://arxiv.org/search/astro-ph?searchtype=author&query=Louis\%2C+T}{\TIB{arXiv}}. Some highlighted ones follow
\begin{enumerate}
\item I have led the power spectrum analysis of the final ACT data release, resulting in \href{https://arxiv.org/abs/2503.14452}{\TIB{this paper}} with over 100 citations, the data release corresponding to this paper can be accessed on \href{https://lambda.gsfc.nasa.gov/product/act/act_dr6.02/}{\TIB{LAMBDA}}.
\item I have studied in details what physical information can be obtained from clusters of galaxies using the next generation of ground based CMB experiments, resulting in the publication of few papers on: \href{https://ui.adsabs.harvard.edu/abs/2017PhRvD..95d3517L}{ \TIB{Cluster lensing}},  \href{https://ui.adsabs.harvard.edu/abs/2016PhRvD..94d3522A}{ \TIB{kSZ effect}}, \href{https://journals.aps.org/prd/abstract/10.1103/PhysRevD.96.123509}{\TIB{pSZ effect}} \\

\end{enumerate}

\newpage


\end{rSection}

\vspace{-0.6cm}


\begin{rSection}{Institutional Responsibilities}
\begin{table}[h]
%%%\centering
{\def\arraystretch{1.5}\tabcolsep=0pt
\begin{tabular}{p{0.15\linewidth}p{0.75\linewidth}}
2024--present &   National Coordinator of the master project "Simons Observatory"   \\
2022--present &   Member of the scientific council DIM Universe  \\
2021--2025 &  Coordinator of the transverse groups "Flavor" and "Cosmology" at IJCLab. \\
2018--2020 &  Organisation of Seminars, Laboratoire de l'Accélérateur Linéaire  \\
2015--2017 &  Founder and co-organiser of Cosmology Coffee, Institut Astrophysique de Paris  \\
2015--2017 &  Postdoc representative for the Laboratory Council, Institut Astrophysique de Paris \\
2013--present &  I am referee for Physical Review D and Physical Review Letters .  
\end{tabular}%
}
\end{table}
\end{rSection}
\vspace{-0.6cm}

\begin{rSection}{Teaching}
\begin{table}[h]
%%%\centering
{\def\arraystretch{1.5}\tabcolsep=0pt
\begin{tabular}{p{0.15\linewidth}p{0.75\linewidth}}
2022--present &   Lectures on "Cosmology of the dark Universe"  Université Paris Saclay. \\
2021--present &   Lectures on "Statistical methods for cosmology"  Euclid summer school. \\
2013--2014 &  Tutorials for the 3rd year course : General Relativity and Cosmology, Oxford University 
\end{tabular}%
}
\end{table}
\end{rSection}
\vspace{-0.6cm}

\begin{rSection}{Outreach}

I actively engage in science communication and public outreach, translating complex cosmological results for a broad audience through press releases, media interviews, and educational events.

\textbf{Press \& Media Coverage}
\begin{itemize}
  \item Interviewed about ACT results in \href{https://www.sciencesetavenir.fr/espace/univers/une-nouvelle-mesure-de-la-constante-de-hubble-confirme-celle-du-satellite-planck_150719}{\TIB{Sciences et Avenir}}.
   \item Interviewed about the strongly lensed SN Refsdal \href{https://www.sciencesetavenir.fr/espace/astrophysique/le-mystere-persistant-du-taux-d-expansion-de-l-univers_171338}{\TIB{Sciences et Avenir}}.
  \item Featured on the national radio station \href{https://media.radiofrance-podcast.net/2021/2/2/NET_MFC_ee6d807d-2c74-49cd-9f1e-7c7d66957f58.mp3}{\TIB{France Culture}} to discuss cosmological findings from ACT.
\end{itemize}

\textbf{Public Communication: Press Releases}
\begin{itemize}
  \item 2025 - Authored the press release on \href{https://www.in2p3.cnrs.fr/fr/cnrsinfo/mesure-de-la-polarisation-du-fond-diffus-cosmologique-par-la-collaboration-act-une-fenetre/}{\TIB{the most precise map of CMB polarization anisotropies}}.
  \item 2024 - Authored the press release on \href{https://www.in2p3.cnrs.fr/en/cnrsinfo/search-echo-big-bang-simons-observatory}{\TIB{the first light of Simons Observatory}}.
  \item 2024 - Authored the press release on \href{https://www.in2p3.cnrs.fr/fr/cnrsinfo/la-collaboration-act-devoile-une-nouvelle-carte-de-la-distribution-de-la-matiere-noire}{\TIB{the distribution of dark matter in the Universe}}.
  \item 2020 - Authored the press release on \href{https://www.ijclab.in2p3.fr/actualite/act-leve-le-voile-sur-lage-de-lunivers/}{\TIB{ACT's measurement of the age of the Universe}}.
\end{itemize}

\textbf{Educational \& Community Outreach}
\begin{itemize}
  \item 2025–present - \textit{“Voyage tout au bout de l'Univers”}: educational presentations on astrophysics in elementary schools.
  \item 2013–present - Participation in numerous public science events, including stargazing at Oxford and “Fête de la science” at the Institut d’Astrophysique de Paris (IAP).
\end{itemize}

\end{rSection}

\begin{rSection}{References }

\begin{enumerate}
\item Prof Joanna Dunkley, jdunkley@princeton.edu
\item Prof Lyman Page, page@princeton.edu
\item Prof Joseph Silk, silk@iap.fr
\item Prof David Spergel, dns@astro.princeton.edu
\end{enumerate}

\end{rSection}

\newpage

\begin{rSection}{Talk and Seminars }
Below is a list of recent talks and seminars 

\begin{table}[h]
%%%\centering
{\def\arraystretch{1.5}\tabcolsep=0pt
\begin{tabular}{p{0.15\linewidth}p{0.75\linewidth}}
 June 2025 & Invited talk at Subatech, Nantes   \\
 June 2025 & Invited talk at the "Cosmology from Home" conference   \\
 June 2025 &  Invited talk at  the "Cosmological Frontiers in Fundamental Physics" conference   \\
 April 2025 & Scientific case of COSMOCal, CNES, Toulouse   \\
 April 2025 &   Invited talk at the "GDR Cophy" conference   \\
 March 2025 &  Talk at IJClab, Orsay   \\
 March 2025 &  Invited talk IAP, Paris   \\
 March 2025 &  I was a speaker at the  \href{https://www.youtube.com/watch?v=sETYHrDBXQg}{\TIB{ACT DR6 data release webinar}} ($>$ 500 attendees).     \\
 Feb 2025 & Invited talk at the "Cosmology on the steep rise" conference   \\
 July 2024 &  Talk on Simons Observatory at the University of Chicago 
\end{tabular}
}
\end{table}

\end{rSection}




\end{document}
